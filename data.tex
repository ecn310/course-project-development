\documentclass[12pt]{article}

% set margins and spacing
\addtolength{\textwidth}{1.3in}
\addtolength{\oddsidemargin}{-.65in} %left margin
\addtolength{\evensidemargin}{-.65in}
\setlength{\textheight}{9in}
\setlength{\topmargin}{-.5in}
\setlength{\headheight}{0.0in}
\setlength{\footskip}{.375in}
\renewcommand{\baselinestretch}{1.0}
\linespread{1.0}

% load miscellaneous packages
\usepackage{csquotes}
\usepackage[american]{babel}
\usepackage[usenames,dvipsnames]{color}
\usepackage{graphicx,amsbsy,amssymb, amsmath, amsthm, MnSymbol,bbding,times, verbatim,bm,pifont,pdfsync,setspace,natbib}

% enable hyperlinks and table of contents
\usepackage[pdftex,
bookmarks=true,
bookmarksnumbered=false,
pdfview=fitH,
bookmarksopen=true,hyperfootnotes=false]{hyperref}

\begin{document}
\title{Development}
% add a fourth name if you have four team members; fill in at least first names below
\author{Lucia Rios-Luy\thanks{lirioslu@syr.edu} \and Meghavarshini Iska\thanks{meiska@syr.edu} \and Sergio Sotelo\thanks{spsotelo@syr.edu}\and Filippo Dona'\thanks{fadona@syr.edu}}
\date{\vskip-.1in \today}
\maketitle

\vskip.3in

\section{Research Question} \label{sec:question}

How has the link between manufacturing employment and economic growth evolved over time and space?

\section{Data Overview} \label{sec:literature}
The World Bank Indicators and the Penn World Table data give us an overview of the different world indicators that estimate the position of each country's economy and development projection. Each data point is supposed to help show how manufacturing employment and economic growth rates have changed over the years using indicators for each country.Two data sets are being used to observe the effects of manufacturing employment on developing and developed countries growth rates. 

\subsection{Data Set 1 (World Bank World Development Indicators Data Bank)}

The World Development Indicators dataset has been sourced from international organizations, national statistical agencies, and multiple surveys that cover social, economic, and environmental indicators for nearly all countries from 1960 to the present day. It includes over a hundred variables, but the most important ones are employment, GDP, and population data, which offers a detailed insight into development trends on a global scale. The number of variables and observations varies depending on the year, with data provided for up to 20,000 country-year combinations across more than one indicator. This data is crucial for policymakers as it offers critical insights and long-term patterns for development across different regions of the world. 

\subsection{Data Set 2 (Penn World Table (PWT) version 10.01)}

 The data for this dataset comes from the Penn World Table (PWT), a comprehensive database compiled by economists Robert C. Feenstra, Robert Inklaar, and Marcel P. Timmer, which is widely used for comparing income, output, input, and productivity across countries. The PWT dataset is generated by the University of Groningen from various public economic reports and national accounts data, making it reliable for macroeconomic analysis. The period from 1950 to 2019 includes observations for 183 countries, providing a global scope that provides data for both developed and developing economies. With over 50 variables, the dataset offers detailed insights into economic indicators such as Real GDP, employment, population, human capital, price levels, and productivity measures. The dataset offers approximately 12,700 country-year observations, making it an invaluable tool for researchers studying economic performance, national productivity, and growth across different regions and time periods.



\section{Data Acquisition}
\label{sec:theory}


    \subsection{Data Acquisition 1 (World Bank World Development Indicators Data Bank)}

\begin{enumerate}
    \item Open the web browser 
       \item  Open the official World Bank World Development Indicators Data Bank page
       \item Clear all selections
       \item Select all countries, all times, and only the variables listed in "Key Variables, Data Set 1", or by opening the Data folder and downloading  "P\_Data\_Extract\_From\_World\_Development\_Indicators (1).xlsx"

       \item Click on the Excel download link to download the .xlsx file
    
\end{enumerate}

    \subsection{Data Acquisition 2 (Penn World Table (PWT) version 10.01)}
\begin{enumerate}
\item Open the web browser 

\item Open the official \href{https://www.rug.nl/ggdc/productivity/pwt/?lang=en}{PWT 10.01} page on the University of Groningen website

\item Click on the Stata download link to download the .dta file

\end{enumerate}
All data is stored in the \href{https://github.com/ecn310/course-project-development/issues/12}{github folder} 



\section{Data Manipulation}
\label{sec:data}

To prepare the data for an accurate analysis, we plan to follow a series of standardized steps on which we can expand on subsequently. Firstly, we import the data in Stata. We then check for outliers using the command "codebook" to verify the integrity of the data, either replacing the missing information with a known value or, if this is not available, deleting the variable from our dataset.  For WDI data, we reshaped it wide to long for easier interpretation and to identify the right values as our variables. 
The data set is integral, in order for us to tabulate certain variables for improved readability and to form initial causal links. We are working with variables, such as GDP, GNI per capita, population growth, total employment, and life expectancy across both data sets. A correlation test will be conducted to compare developing to developed countries in both data sets, and we will graph our two main variables, GNI per capita and manufacturing employment. In a comparative analysis, we would use "codebook" to establish how often a variable is present in a dataset, and also to be able to derive and analyze the mean, median, and standard variation.


\href{https://github.com/ecn310/course-project-development/blob/main/WDI.do}{Data 1 do-file} 

\href{https://github.com/ecn310/course-project-development/blob/main/Penn%20World%20Table%20(PWT)%20version%2010.01.do}{Data 2 do-file} 

Specifically to WDI data, we must first deselect all, and then select to include all countries across all available years, and only the variables listed in the Key variables listed below for Data Set 1: GNI per capita, PPP (current international in USD) - Ratio Scale, Population growth: annual percent -  Ratio Scale,  Life expectancy - Nominal/Ratio Scale. Once the Excel file has been downloaded, we import it into Stata and proceed to reshape it for easier interpretation. Steps to follow to reshape the data can be found here: \href{https://github.com/ecn310/course-project-development/blob/main/WDI.do}{Data 1 do-file}.
Once the data has been reshaped, it can be downloaded as a dta file. Note that should you wish to include more variables, the steps to follow in Stata remain largely unaltered, but other variables must be selected prior to downloading the spreadsheet from the WDI website. 


\section{Linking Datasets}
\label{sec:discussion}

We will merge the World Bank World Development Indicators, which contains population growth data for each country, with Dataset 2, which includes real GDP output-side. Merging these two Datasets allows us to analyze and calculate economic growth by linking GDP figures to demographic trends across various years and regions. In Dataset 1 and Dataset 2 the relevant variables to merge both of these Datasets together are “country name”, and "year". 


\section{Key Variables}
\label{sec:result}


\subsection{Data Set 1 (World Bank: World Development Indicators)}

\begin{itemize}
    \item GNI per capita, PPP (current international in USD) - Ratio Scale 
    \item Population growth: annual percent -  Ratio Scale 
    \item Life expectancy: low mortality rate can aid in employment - Nominal/Ratio Scale 
\end{itemize}

\subsection{Data Set 2 (Penn World Table (PWT) version 10.01)}
\begin{itemize}
   \item rgdpo: Real GDP output-side, used to measure economic growth - Ratio Scale
   \item emp: Total employment, providing insight into labor market trends. - Ratio Scale 
   \item pop: Population, to calculate per capita measures like GDP per capita. - Ratio Scale
\end{itemize}


\end{document}
